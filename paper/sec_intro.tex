% vim:syntax=tex
%

Software developers are often confronted with maintenance tasks that involve navigation of repositories that preserve vast amounts of project history.
 Navigating these software repositories can be a time-consuming task, because their organization can be difficult to understand.
A software developer who is tasked with changing a large software system
spends effort on program comprehension activities to gain the knowledge
needed to make the change~\cite{Corbi:1989}.

 Fortunately, topic models such as latent semantic indexing (LSI)~\cite{Deerwester:1990} and latent Dirichlet allocation (LDA)~\cite{Blei-etal:2003} can help developers to navigate and understand software repositories by discovering topics (word distributions) that reveal the thematic structure of the data~\cite{Linstead-etal:2007,Thomas-etal:2011,Hindle_etal:2012}.


When modeling a source code repository, the corpus typically represents a snapshot of the code.
That is, a topic model is often trained on a corpus that contains documents that represent files from a particular version of the software.
Keeping such a model up-to-date is expensive, because the frequency and scope of source code changes necessitate retraining the model on the updated corpus.
However, it may be possible to automate certain maintenance tasks without a model of the complete source code.
For example, when a developer is assigned to a task, a topic model can be used
to locate features within the source code related to this task.
In this scenario, a model of the changesets may be more useful than a model of the files,
as the model will be up-to-date.
Moreover, since topic modeling techniques like LSI and LDA are now online,
a topic can be updated in real-time as work is being completed.

This paper makes the following contributions:

\begin{itemize}
    \item An approach to modeling changesets for feature location
    \item An approach to evaluating a feature location technique temporally
    \item A tool that implements the approaches
    \item A case study of the of the two approaches 
    \item An updated dataset of systems used in this study
    \item Towards increasing open science principles in SE,
        source code and data for replication of this study
\end{itemize}

We first review related work (\S\ref{sec:related}).
We next describe our methodolgy (\S\ref{sec:methodology}) and case study (\S\ref{sec:study}),
which spans fourteen open source Java systems.
We then conclude (\S\ref{sec:conclusion}).
