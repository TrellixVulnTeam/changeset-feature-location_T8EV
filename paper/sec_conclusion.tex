% vim:syntax=tex

In this paper we conducted a study on modeling the topics of changesets in comparison to the traditional snapshot approach.
We use latent Dirichlet allocation (LDA) to extract linguistic topics from
changesets and snapshots (releases).

We addressed two research questions regarding the topic modeling of changesets.
First, we compare a batch topic-modeling-based FLT trained on the changesets
of a project's history to one trained on the snapshot of source code entities.
Second, we compare a batch topic-modeling-based FLT trained on changesets
to a temporal topic-modeling-based FLT trained on the same changesets over time.
We found that changesets can perform as well as or better than snapshots.
We also show that temporal analysis more accurately portrays how a FLT would execute in a real environment.


Future work includes deploying this appoach in a development environment.
Since the source to our approach is online, we encourage other researchers
to investigate this future work as well.
We also would like to expand the temporal parts of this study to include
both snapshots and changesets.
It would be particularly useful to compare results between batch snapshots and temporal snapshots.
Additional future work includes expanding our study to other systems,
particularly ones that are not Java.
It seems unlikely that our results are specific to Java systems,
though we cannot confirm this assumption without experimentation.


