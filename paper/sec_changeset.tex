% vim:syntax=tex

In this section we describe how a TM-based FLT can use changesets.

\subsection{Terminology}

In addition to the terminology described in Section~\ref{sec:related}, we use
the following terminology to describe the document extraction and retrieval
process of changesets.

A \textit{diff} is a set of text which represents the differences between two
texts. A \textit{patch} is a set of instructions (i.e., diffs) that is used to
transform one set of texts into another. \textit{Context lines} denote text
useful for transforming the text, but do not represent the differences.
\textit{Added lines} are lines which were added in order to transform the first
text into the second. Similarly, \textit{removed lines} are lines which are
removed for this same purpose. A \textit{changeset}, ideally, represents a
single feature modification, addition, or deletion, which may crosscut many
source code entities. A \textit{commit} is a representation of a changeset in a
version control system, such as Git or Subversion. Figure~\ref{fig:diff} shows
an example changeset from Git.


\begin{figure*}[t]
\centering
\footnotesize
\begin{lstlisting}[language=diff, basicstyle=\ttfamily]
diff --git a/src/java/net/sf/jabref/EntryEditor.java b/src/java/net/sf/jabref/EntryEditor.java
index 8c56723..6b4788e 100644
--- a/src/java/net/sf/jabref/EntryEditor.java
+++ b/src/java/net/sf/jabref/EntryEditor.java
@@ -669,7 +669,8 @@ public class EntryEditor extends JPanel implements VetoableChangeListener {
     public void storeCurrentEdit() {
         Component comp = Globals.focusListener.getFocused();
         if ((comp == source) || ((comp instanceof FieldEditor) && this.isAncestorOf(comp))) {
-            ((FieldEditor)comp).clearAutoCompleteSuggestion();
+            if (comp instanceof FieldEditor)
+                ((FieldEditor)comp).clearAutoCompleteSuggestion();
             storeFieldAction.actionPerformed(new ActionEvent(comp, 0, ""));
         }
     }
\end{lstlisting}
\caption{Example of a \texttt{git diff}.
This changeset addresses JabRef's Issue \#2904968.
Black or blue lines denote metadata about the change useful for patching.
In particular, black lines represent context lines (beginning with a single space).
Red lines (beginning with a single~\texttt{-}) denote line removals,
and green lines (beginning with a single~\texttt{+}) denote line additions.}
\label{fig:diff}
\vspace{-10pt}
\end{figure*}

\subsection{Feature location using changesets}

The overall difference in our methodology and the standard methodology described
in Section~\ref{sec:snapshot-flt} is minimal. For example, compare
Figures~\ref{fig:snapshot}~and~\ref{fig:changeset}. In the changeset approach,
we only need to replace the documents on which the topic model is trained while
the remainder of the approach remains the same.

The changeset approach requires two types of document extraction: the snapshot
of the state of source code at a commit of interest, such as a tagged release,
and every changeset in the source code history leading up to the same commit of
interest. The left side of Figure~\ref{fig:changeset} illustrates the
dual-document extraction approach.

The document extraction process for the snapshot remains the same as covered in
Section~\ref{sec:related} while the document extractor for the changesets parses
each changeset for the removed, added, and context lines. From there, each line
is tokenized by the text extractor. The same preprocessor transformations as
before occur in both the snapshot and changesets. The snapshot vocabulary is
always a subset of the changeset vocabulary~\cite{Corley-etal_2014}.

The right side of Figure~\ref{fig:changeset} illustrates the retrieval process.
The key intuition to our methodology is that a topic model such as LDA or LSI
can infer \emph{any} document's topic proportions regardless of the documents
used to train the model. In fact, this is also what determining the topic
proportions of a user-created query relies on. Likewise, so are other unseen
documents. In our approach, the seen documents are changesets and the unseen
documents are the source code entities of the snapshot.

Hence, we train a topic model on the changeset corpus and use the model to index
the snapshot corpus. Note that we never construct an index of the changeset
documents on which the model is trained, nor do we use the snapshot corpus
during training. In our approach, we only use the changesets to continuously
update the topic model and only use the snapshot for indexing.

\todo{r3: found this confusing}
To leverage the online functionality of the topic models, we can also intermix
the model training, indexing, and retrieval steps. First, we initialize a model
in online mode. Then, as changes are made, the model is updated with the new
changesets as they are committed. That is, with changesets, we incrementally
update a model and can query it at any moment. Our historical simulation
(\S~\ref{sec:methodology}) relies on this insight.

\subsection{Why changesets?}

We choose to train the model on changesets, rather than another source of
information, because they also represent what we are primarily interested in:
program features. A single changeset provides text of an addition, removal, or
modification of a single feature. A developer can to some degree comprehend what
a changeset accomplishes by examining it, such as during a code review, much
like examining a source file directly.

While a snapshot corpus has documents that represent a program, a changeset
corpus has documents that represent programming. If we consider every changeset
affecting a particular source code entity, then we gain a sliding-window view of
that source code entity over time and the contexts those changes were performed
in. This is akin to summarizing code snippets with machine
learning\cite{Ying-Robillard_2013}, where in our case a changeset gives a
snippet-like view of the code required to complete a task. For example, in
Figure~\ref{fig:diff}, we can see the entire method being changed when the
context lines are considered.

Additionally, Vasa et al.~\cite{Vasa-etal_2007} observe that code rarely changes
as software evolves. The implication is that the topic modeler will see
changesets containing the same source code entity only a few times, perhaps only
once. Since topic modeling a snapshot only sees an entity once, topic modeling a
changeset can miss no information.

Using changesets also implies that the topic model may gain some noisy
information from these additional documents, especially removals. However, Vasa
et al.\ also observe that code is less likely to be removed than it is to be
changed. This implies that the noisy information would likely remain in both
snapshot-based models and changeset-based models.

Indeed, it appears desirable to remove changesets from the model that are old
and no longer relevant to the current snapshot of the system. There would be no
need for this because online LDA already contains features for increasing the
influence newer documents have on the model, thereby decaying the effect of the
older documents on the model.


